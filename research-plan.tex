\chapter{Research Plan}
	
	Outline what work I hope to do with more detail on the near future.
	
	Gantt?
	
	\section{SpiNNaker Modelling}
		
		In order to experiment with the way new interconnection options affect
		SpiNNaker an accurate model is required. Will work with Mikel and Javier to
		extend the SpiNNaker simulator used in the original paper to include models
		of inter-board links.
		
		How long will this take?
		
		\subsection{Performance Benchmarking}
			
			Part of the aim of the work is to produce a comparison between this
			software model, an FPGA model and a prototype system. Successful
			comparison of the three will allow higher certainty in results obtained
			from other work (namely the interconnect experiments to follow).
		
		\subsection{Interconnect Experiments}
			
			The model will be further extended to simulate new links between boards
			and performance again tested.
			
				\subsubsection{Topologies}
					
					Topologies to test include: express cubes, small-world, \ldots
				
				\subsubsection{Routing}
					
					One major consideration when changing the topology is to create
					appropriate routing schemes for each topology. In particular an
					analysis of their use of routing table resource would be important.
		
	
	\section{Effects of Multicast}
		
		Multicast networks are unusual in many respects and complicate things
		somewhat. The model should be further extended to cope with multicast
		traffic and further work carried out in this aim.
		
		How long will this take?
	
	\section{`SpiNNaker 2' Architecture}
		
		It is likely that a new version of SpiNNaker will be designed with some
		funding coming in soon. Based on the research into the interconnect here,
		would an alternative topology be appropriate?
		
		\subsection{Types of Transmission}
			
			Switching away from Silistix (they went bye-bye!), alternative
			technologies, e.g. high-speed serial, might be used to link chips
			together. Such technologies may have very different cost structures and
			call for new network topologies. Given we currently have concentrated
			multiplexed links, why not have the same for stuff lower down?
		
		\subsection{Multiple Networks}
			
			SpiNNaker's interconnect is obviously heavily targeted at neural
			simulation but at the expense of other activities. For example, set-up and
			management are a bit of a pain. Also, other algorithms don't suit it so
			well. It is possible SpiNNaker will benefit from having a second network
			for these purposes?

