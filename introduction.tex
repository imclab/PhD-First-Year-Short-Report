\chapter{Introduction}
	
	Some high-level motivation and context to sort of set the scene before going
	into what other folk have done and what I'm hoping to do.
	
	\section{Parallel Computing}
		
		The race to drive up the speed of individual processor cores ground to a
		halt in the mid 2000s leaving designers pushed to find a use for the vast
		number of transistors made available by Moore's law. In the years that
		followed we've seen multi-core processors appearing in everything from
		mobile phones to large super computers.
		
		\subsection{Programming}
			
			Hard to program parallel systems - deadlocks, locality becomes an issue --
			things can be a long way away. Cache tries to hide this but fails to
			scale. Other approaches might work (e.g. spinnaker has slow step with
			communication happening before the next step, i.e. synchronous model).
		
		\subsection{Architecture}
			
			Lots of computers must be connected together. Current styles are largely
			regular. Each offers its own advantages to different tasks. More in
			background chapter.
	
	\section{Computational Problems}
		
		Computers are faced with a wide range of problems all of which require
		different patterns of communication between the different parts of the
		program influencing the type of interconnect topology they suit.
		
		\subsection{TODO: More Examples}
		
		\subsection{Gossiping Algorithms}
			
			Generally lots of broadcast global messaging, sometimes with low
			reliability requirements.
		
		\subsection{Fluid Dynamics}
			
			Such algorithms such as heat propagation in a material. Lots of
			computation and very local, neighbour-to-neighbour communication.
		
		\subsection{Brain Simulation}
			
			% TODO: Cite overview of computational neuroscience?
			
			The brain is an extremely powerful computer about which little is
			understood. The field of computational neuroscience hopes to bring
			understanding of the computational abilities and mechanisms of the
			brain.
			
			One approach to this problem is to try and produce simulations of models
			of the brain in order to understand and study their behaviour. These
			approaches often take the form of simulated populations of neurons, one
			of the basic building-blocks of the brain. Such models typically consist
			of a set of neurons, each connected to many other neurons.  Unlike
			digital circuits where each individual component connects to only a few
			or even one other component, neurons tend to be connected to hundreds or
			thousands of other neurons.
			
			Simulations of the brain therefore present a computational challenge as
			large amounts of communication must take place between all the neurons
			in the simulated system. Various architectures have been designed to
			solve this task and a selection are presented in Chapter
			\ref{chap:background}.
			
			TODO: What is going on in brain simulators?
	
	\section{Architecture}
		
		Two major considerations when designing an architecture are the topology and
		way in which packets will be routed around the system.
		
		\subsection{Topology}
			
			Again?
		
		\subsection{Routing}
			
			Again?
		
		\subsection{Multicast}
			
			Quite simple, really.

