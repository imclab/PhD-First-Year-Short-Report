\chapter{Conclusion}
	
	% Brain simulation is obviously a valuable tool for research into the brain and
	% medicine, both for small detailed and large simple models.
	
	Modelling the brain has great potential to allow understanding of both its
	mechanism and medical treatment. Unfortunately, despite its potentially
	significant scientific impact, the computational tools for simulating these
	models is still limited due to their unconventional balance of light
	computation and heavy communication.
	
	% Sadly, super computers are not cut out for the job for large simulations. Too
	% much focus on computation. Various approaches tried but currently most
	% promising are digital.
	
	Conventional super computers continue to push the envelope in raw
	computational power but have been comparatively slow to advance the level of
	communication possible between parts of the system. As a result, neural
	simulations running on super computers have been forced to work with small
	models hundreds of thousands of times smaller than that of the human brain.
	
	Numerous alternative architectures have been proposed, many of which are
	highly specialised to simulations of a particular style of neural model.
	Though achieving impressive performance, such devices often prove difficult to
	configure either due to the challenges of working with analog components or
	the low-level nature of FPGA development.
	
	% SpiNNaker is a very promising arch but its interconnect not well suited to all
	% of the job and better can be done as preliminary work has hinted. In
	% particular, new generations will not be able to rely on the same technology
	% and will need to look to high-speed serial which may mean new topologies.
	
	The SpiNNaker architecture brings a great deal of flexibility for
	communication-heavy neural network models. Preliminary studies have shown that
	improvements to the SpiNNaker architecture are possible such as the addition
	of a small number of random links to the SpiNNaker network which may be able
	to significantly reduce the distance neural spikes must travel to reach other
	neurons in the system.
	
	% This work will aim to develop a new architecture based on lessons learned from
	% SpiNNaker. It will focus on modelling interconnect both of SpiNNaker and of
	% future machines.
	
	It has been noted that the technology used by the current generation of
	SpiNNaker to connect chips together will not be available for future systems.
	The continued development of high speed serial technology presents a possible
	alternative with potentially much greater performance but with it comes
	differing costs in chip resource.
	
	In addition, Moore's law continues to provide ever greater resources, the use
	of which must be decided in the design of a new neural network architecture.
	
	% Work on the models and computational bit are not the focus but also offer
	% interesting areas of research for others or future work. This focus is a good
	% idea because communication heavy stuff relies on the interconnect above all.
