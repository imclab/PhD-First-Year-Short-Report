\chapter{Introduction}
	
	% %Some high-level motivation and context to sort of set the scene before going
	% %into what other folk have done and what I'm hoping to do.
	% 
	% Modern computer systems represent some of the most complex devices ever
	% constructed. Indeed, current computer technologies have enabled everything
	% from global, instantaneous communications via the internet to faster and more
	% effective cancer treatments\cite{nassif}. Despite this, the brain still
	% outperforms conventional digital computers at many tasks.
	% 
	% Considerable effort has been made by researchers to understand how the brain
	% works. The small-scale operation of individual neurons in the brain is now
	% relatively well understood but the manner of their interactions is still not
	% clear. Current efforts to understand the brain attempt to model it as a
	% biological computer. Such biological computers differ greatly from digital
	% computers and so simulating their behaviour using existing technology is
	% challenging.
	% 
	% Recent neural models such as the Spaun brain simulator \cite{eliasmith12}
	% exhibit a remarkable range of cognitive abilities such as memory, problem
	% solving and pattern recognition. As well as realistic functional behaviour,
	% they also respond to failures, such as neuron death, just as real brains do
	% exhibiting a gradual decline in functionality. The existence of realistic
	% models means studying the properties of the brain in a highly controlled,
	% observable manner as in a simulator is becoming a possibility.
	
	Modern computer systems represent some of the most complex devices ever
	constructed. Indeed, current computer technologies have enabled everything
	from global, instantaneous communications via the internet to faster and more
	effective cancer treatments\cite{nassif}. Despite this, the brain still
	outperforms conventional digital computers at many tasks.
	
	Considerable effort has been made by researchers to understand how the brain
	works. The small-scale operation of individual neurons in the brain is now
	relatively well understood but the manner of their interactions is still not
	clear. By modelling the behaviour of networks of neurons it is possible to
	gain a better understanding of their collective behaviour. Such models are
	challenging to fit into modern computer architectures since the vast
	parallelism available to individual neurons in a brain is in sharp contrast
	with the computing resources available today.

	In this introduction I hope to elaborate on the role that modelling has on
	understanding the brain and on the work being done to build machines to fulfil
	this task. I also outline the specific aspect of these machines on which my
	work will focus, the interconnection network, and its importance in achieving
	the wider goal.
	
	\section{Why Model The Brain?}
	
		
		%% TODO REWORD BELOW
		
		Cutting-edge neural models, such as Spaun \cite{eliasmith12}, are able to
		demonstrate a remarkable range of cognitive abilities such as memory,
		problem solving and pattern recognition. Notably, Spaun exhibits very
		similar behaviour to humans in terms of the way its memory degrades over
		time. It also suffers the same gradual reduction in function as random
		neurons die off. This level of realism means that experiments which would
		not be possible may become feasible using the Spaun model.
		
		In addition to Spaun, many other neural models have emerged. In distinct
		contrast to Spaun, researchers on the Blue Brain project \cite{markram06}
		have focused on developing high-accuracy simulations of much smaller
		collections of neurons. Such models are highly biologically plausible but
		do not result in the complex, high-level behaviour shown by simulators such
		as Spaun.
		
		Unfortunately these models present a great challenge to modern computer
		systems. The Spaun simulator takes around two and a half hours of compute
		time to simulate one second of neural activity on a high-end workstation
		computer. The Blue Brain project requires a commercial super-computer to
		reach simulation speeds still around an order of magnitude slower than
		biological real-time. The basic mechanism of these neural simulators, and
		their computational challenges is described in further detail in
		\S\ref{sec:simulating-brains}.
	
	\section{Conventional Approaches}
	
		% Why not use a big ass-computer? Well big computers are good for computation
		% but neural stuff is all about communication. Not so great. As mentioned
		% there is another way 
		
		Conventional super computers are designed to provide immense computational
		power performing quadrillions ($10^{15}$) of calculations to be performed
		per second. Despite these extremely impressive feats of computation, these
		machines often feature substantially less impressive interconnection
		networks tying them together. This balance of computation and communication
		is contrary to the brain whose neurons represent relatively little
		computational power but are extraordinarily well connected.
		
		The Blue Brain project is notable in its use of conventional super
		computers. Though they are severely limited in the size of their models,
		only around 100,000 neurons compared with almost 100 billion in the human
		brain, they are able to precisely simulate the detailed biological and
		chemical processes occurring within each neuron.
		
		\S\ref{sec:super-computers} goes into greater detail on current super
		computer architectures and their strengths and weaknesses.
	
	\section{Why Interconnect Matters}
		
		My research focuses on the challenge of developing architectures designed
		for the simulation of large-scale networks, such as Spaun, which are heavily
		communication bound.
		
		The SpiNNaker project \cite{furber06} has developed a novel computer
		architecture which emphasises communication over computation based on a
		network of over one million small, low-power processors. The purpose built
		interconnection network is designed to handle neural simulations of up to
		one billion neurons running in biological real-time. A detailed introduction
		to the SpiNNaker architecture is given in \S\ref{sec:spinnaker}.
		
		SpiNNaker's interconnection network is designed to allow efficient
		communication of signals produced by the simulated neurons which typically
		only need to travel short distances in the system. As the system scales, new
		types of interconnection network have been introduced in order to keep the
		wiring in the system practical. \S\ref{sec:high-speed-serial} describes the
		technology used to simplify the system's wiring. Preliminary work, described
		in \S\ref{sec:interconnect-modelling}, has been conducted to asses the
		impact of these changes for neural network simulation. It shows the impact
		changes in the system's interconnect can have on the time taken by neural
		signals to move through the system.
		
		The topic of practical issues when designing interconnect issues is studied
		in \S\ref{sec:wiring-up-large-spinnaker-machines} where the task of wiring
		up large SpiNNaker systems consisting of 1,200 of connected circuit boards
		is tackled. This is followed up by the development of an unconventional
		semi-random interconnection network which takes into account these wiring
		considerations in \S\ref{sec:small-world-super-computers}. This design of
		network has the potential to improve the performance of SpiNNaker's
		architecture with a minimal cost in increased cabling requirements.
	
		% SpiNNaker is a neural simulator emphasising communication over computation
		% by using lots of low-powered cores. It hopes to do real-time simulation. It
		% uses a very straight-forward mesh interconnect which suffers very long
		% latencies for non-local traffic. Alternative topologies can overcome this.
	
		My work hopes to develop a new architecture with a focus on the
		interconnection network, a key aspect of a machine targeted at communication
		bound problems such as neural simulation. Section \S\ref{sec:research-plan}
		outlines the plan of research I intend to take to reach this goal following
		on from my preliminary studies.
