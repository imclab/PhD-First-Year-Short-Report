\chapter{Research Plan}
	
	In this chapter an outline of the work to be carried out is presented. Due to
	the embryonic stage of the research, later work is presented with less detail.
	Figure \ref{fig:plan-gantt} shows a Gantt chart containing the key stages of
	the project which are outlined in chronological order in the sections below.
	
	\begin{figure}[b!]
		\center
		\begin{tikzpicture}[thick,x=0.25cm]

%%%%%%%%%%%%%%%%%%%%%%%%%%%%%%%%%%%%%%%%%%%%%%%%%%%%%%%%%%%%%%%%%%%%%%%%%%%%%%%%
% Hacked-up Gantt Library
%%%%%%%%%%%%%%%%%%%%%%%%%%%%%%%%%%%%%%%%%%%%%%%%%%%%%%%%%%%%%%%%%%%%%%%%%%%%%%%%

% An entry in the Gantt chart. Takes a label, start offset, length and slack.
% Also defines a pair of labels "[label] start" and "[label] end" which can be
% used for drawing dependency lines.
\newcommand{\ganttEntry}[4]{
	% Label
	\node (label)
				[below=1.5ex of label.south east,anchor=east,minimum height=1.7em]
				{#1}
				;
	\coordinate (gantt labels end) at (label.south east);
	
	% Box
	\draw ([shift={(#2   ,-.9ex)}]label.north east) rectangle
	      ([shift={(#2+#3,0.9ex)}]label.south east);
	
	% The tips of the box
	\coordinate (#1 end)
	         at ([shift={(#2+#3,0.9ex)}]label.south east);
	\coordinate (#1 start)
	         at ([shift={(#2   ,-.9ex)}]label.north east);
	
	% Slack line
	\draw [ultra thick]
	      ([shift={(#2+#3,0)}]$(label.north east)!0.5!(label.south east)$)
	   -- ++(#4,0);
}

\newcommand{\ganttDep}[2]{
	\draw [->,red] (#1 end) -| (#2 start);
}

\newcommand{\ganttVSep}[2]{
	\draw [#2] ([shift={(#1,0)}]gantt labels start) -- ([shift={(#1,0)}]gantt labels end);
}

% A new gantt chart. Takes a list of x-offset/label/major-label tuples. For each
% tuple a line is created with x-offset from the previous line and the span is
% labelled with "label". If major-label given, a major label will be drawn
% centered over the previous entries up until the last major-label.
\newenvironment{gantt}[1]{
	% Start the list of labels
	\node (label) [white] {Ag};
	\coordinate (gantt labels start) at (label.north east);
	\def\periods{#1}
}{
	\begin{scope}[on background layer]
		% Thick line separating from labels
		\draw (gantt labels start) -- (gantt labels end);
		
		% Start of the area covered by a "major" label
		\coordinate (gantt maj label start) at (gantt labels start);
		
		\foreach \x/\lab/\mlab in \periods {
			\coordinate (next gantt labels start) at ([shift={(\x,0)}]gantt labels start);
			\coordinate (next gantt labels end)   at ([shift={(\x,0)}]gantt labels end);
			
			% Minor label
			\node at ($(gantt labels start) !0.5! (next gantt labels start)$)
			      [anchor=west,rotate=90]
			      {\lab}
			      ;
			
			% Separator
			\ifthenelse{\equal{\mlab}{}}{
				\draw [help lines] (next gantt labels start) -- (next gantt labels end);
			}{
				\draw [help lines,thick] (next gantt labels start) -- (next gantt labels end);
			}
			
			\coordinate (gantt labels start) at (next gantt labels start);
			\coordinate (gantt labels end)   at (next gantt labels end);
			
			% Major label
			\ifthenelse{\equal{\mlab}{}}{}{
				\coordinate (next gantt maj label start) at (gantt labels start);
				\node at ($(gantt maj label start) !0.5! (next gantt maj label start)$)
				      [yshift=1cm,anchor=south]
				      {\mlab}
				      ;
				\coordinate (gantt maj label start) at (next gantt maj label start);
			}
		}
	\end{scope}
}



%%%%%%%%%%%%%%%%%%%%%%%%%%%%%%%%%%%%%%%%%%%%%%%%%%%%%%%%%%%%%%%%%%%%%%%%%%%%%%%%
% The Chart...
%%%%%%%%%%%%%%%%%%%%%%%%%%%%%%%%%%%%%%%%%%%%%%%%%%%%%%%%%%%%%%%%%%%%%%%%%%%%%%%%

\begin{gantt}{
	4/Aug/, 4/Sep/, 4/Oct/, 4/Nov/, 4/Dec/2013,%
	2/Q1/, 2/Q2/, 2/Q3/, 2/Q4/2014,%
	2/Q1/, 2/Q2/, 2/Q3/, 2/Q4/2015,%
	2/Q1/, 2/Q2/2016%
}
	\ganttEntry{SpiNNaker Modelling}          {0}{5}{3}
	\ganttEntry{Small-World SpiNNaker}        {5}{4}{1}
	\ganttEntry{Topology Comparison}          {8}{8}{4}
	\ganttEntry{Place and Routeability}       {16}{4}{0.5}
	\ganttEntry{Multicast Simulation}         {20}{1}{0.5}
	\ganttEntry{Interconnect Evaluation}      {21}{1}{0.5}
	\ganttEntry{Architecture Design}          {22}{6}{4}
	\ganttEntry{Architecture Evaluation}      {26}{4}{2}
	\ganttEntry{Thesis Writing}               {30}{8}{2}
	
	\ganttDep{SpiNNaker Modelling}{Small-World SpiNNaker};
	\ganttDep{SpiNNaker Modelling}{Topology Comparison};
	\ganttDep{SpiNNaker Modelling}{Place and Routeability};
	
	\ganttDep{Place and Routeability}{Multicast Simulation};
	
	\ganttDep{Place and Routeability}{Architecture Design};
\end{gantt}

\end{tikzpicture}

		
		\caption[Gantt chart of proposed research plan]{Gantt chart of proposed
		research plan. Arrows indicate dependencies, thick lines indicate slack.
		Note non-linear scale.}
		\label{fig:plan-gantt}
	\end{figure}
	
	
	\section{SpiNNaker Modelling}
		
		% In order to experiment with the way new interconnection options affect
		% SpiNNaker an accurate model is required. Will work with Mikel and Javier to
		% extend the SpiNNaker simulator used in the original paper to include models
		% of inter-board links.
		%
		% How long will this take?
		
		% TODO: What journal?
		
		As part of the ongoing SpiNNaker project a study is being carried out to
		compare how network simulations compare with the real hardware. As part of
		this work, researchers have built an FPGA accelerated model of SpiNNaker's
		interconnection network. The work hopes to verify the accuracy of the model
		against both a traditional software model and also the final silicon
		implementation. The results of this work are intended for a Journal
		publication aiming for submission around the end of September 2013.
		
		My participation in this project will be to develop the software simulation
		to model and evaluate the network implemented in SpiNNaker. Once built, this
		simulator will be valuable for my own research into alternative network
		topologies.  This section outlines in further detail the motivation and
		requirements for the simulator followed by a discussion of its applications
		for my own work.
		
		\subsection{Software Simulator Limitations}
		
			The prototype software model is built on the INSEE simulator
			\cite{navaridas11insee} which is designed to simulate a wide variety of
			networks. Unfortunately, INSEE's router model is different from that found
			in SpiNNaker and the FPGA model. The differences in their behaviour are
			outlined below.
			
			% TODO: elaborate
			
			SpiNNaker makes use of store-and-forward routing where a message must be
			fully received before it can be routed to the next point in its path. By
			contrast, INSEE is designed to support cut-through routing where once the
			first part of a message has been received it may immediately begin the
			routing process.
			
			The other key difference between INSEE and the actual SpiNNaker
			architecture is the way packets from incoming links are arbitrated. Figure
			\ref{fig:arbitration} shows the arbitration schemes for INSEE and
			SpiNNaker. INSEE uses a simple round-robin arbitration scheme while
			SpiNNaker (and the FPGA model) use a tree of arbitrators\footnote{The
			bandwidth available at each level of the arbitration tree scales with the
			maximum input bandwidth for each level.}.
			
			\begin{figure}
				\begin{subfigure}[t]{0.50\textwidth}
					\center
					\begin{tikzpicture}[thick]
	
	\node (arbitrator)
		[draw,minimum height=6cm]
		{
			\tikz[minimum height=0cm]
				\node[rotate=90]{RR};
		};
	
	\newcommand{\inputline}[2]{
		\draw [<-]
		      ($(arbitrator.north west) !#1! (arbitrator.south west)$) -- ++(-0.5,0)
		      node [left] {#2}
		      ;
	}
	\inputline{0.1}{N}
	\inputline{0.2}{S}
	\inputline{0.3}{E}
	\inputline{0.4}{W}
	\inputline{0.5}{NW}
	\inputline{0.6}{SE}
	\inputline{0.7}{P1}
	\inputline{0.8}{\ldots}
	\inputline{0.9}{P18}
	
	\draw [->]
	      ($(arbitrator.north east) !0.5! (arbitrator.south east)$) -- ++(0.5,0)
	      node [right] {Router}
	      ;
	
\end{tikzpicture}

					
					\caption{INSEE}
					\label{fig:arbitrationINSEE}
				\end{subfigure}
				\begin{subfigure}[t]{0.50\textwidth}
					\center
					\begin{tikzpicture}[thick]
	
	\newcommand{\rootarb}[3]{
		\node (#1)
			[draw,minimum height=#2] #3
			{
				\tikz[minimum height=0cm]
					\node[rotate=90]{RR};
			};
	}
	
	\newcommand{\arb}[4]{
		\rootarb{#1}{#2}{[left=0.5 of $(#4.north west) !#3! (#4.south west)$]}
		\draw [->] (#1) to (#4);
	}
	
	\rootarb{root arb}{4cm}{}
	
	\arb{nsew arb}{ 2.0cm}{0.1}{root arb}
	\arb{neswc arb}{2.0cm}{0.9}{root arb}
	
	\arb{ns arb}{  1cm}{0.1}{nsew arb}
	\arb{ew arb}{  1cm}{0.9}{nsew arb}
	\arb{nesw arb}{1cm}{0.1}{neswc arb}
	\arb{core arb}{1cm}{0.9}{neswc arb}
	
	\newcommand{\inputline}[3]{
		\draw [<-]
		      ($(#2.north west) !#1! (#2.south west)$) -- ++(-0.5,0)
		      node [left] {#3}
		      ;
	}
	\inputline{0.25}{ns arb}{N}
	\inputline{0.75}{ns arb}{S}
	
	\inputline{0.25}{ew arb}{E}
	\inputline{0.75}{ew arb}{W}
	
	\inputline{0.25}{nesw arb}{NW}
	\inputline{0.75}{nesw arb}{SE}
	
	\inputline{0.25}{core arb}{P1}
	\inputline{0.50}{core arb}{\ldots}
	\inputline{0.75}{core arb}{P18}
	
	\draw [->]
	      ($(root arb.north east) !0.5! (root arb.south east)$) -- ++(0.5,0)
	      node [right] {Router}
	      ;
	
\end{tikzpicture}

					
					\caption{SpiNNaker}
					\label{fig:arbitrationSpiNNaker}
				\end{subfigure}
				
				\caption[Incoming packet arbitration schemes in INSEE and
				SpiNNaker]{Incoming packet arbitration schemes in INSEE and SpiNNaker.
				Boxes marked `RR' are round-robin arbitrators, P1-P18 are connections to
				local processor cores.}
				\label{fig:arbitration}
			\end{figure}
			
			This difference here extends beyond the order in which contesting requests
			will be serviced. In INSEE, each input has a buffer associated with it
			from which the router will extract messages and forward them to the input
			buffer of the next node, one per simulated cycle. In the SpiNNaker design,
			messages are buffered at each level of the tree and eventually placed in a
			pipeline within the router (equivalent to further buffering) and an output
			buffer. The interaction of all these buffers is not modelled by INSEE and
			thus the results produced are less well matched with the actual SpiNNaker
			system.
			
			% Part of the aim of the work is to produce a comparison between this
			% software model, an FPGA model and a prototype system. Successful
			% comparison of the three will allow higher certainty in results obtained
			% from other work (namely the interconnect experiments to follow).
		
		\subsection{Simulator Improvement Plan}
			
			Given the limitations of INSEE mentioned in the previous section, two
			possible approaches must be considered. Either INSEE must be modified to
			incorporate a more realistic model of the router or an alternative
			simulator must be used.  One important factor in the decision is the
			`ramp-up' time required to gain familiarity with the INSEE code-base
			compared with the time required to develop or extend an alternative
			simulator. The other factor is the utility of the simulator for my own
			research.
			
			Since INSEE is an established tool which has been used in similar work it
			is potentially a strong choice. In order to determine its suitability for
			this work and my own research a small amount of time will be initially
			spent analysing its design. If it is found to be suitable development of
			an extended version will commence.
			
			A possible alternative to INSEE is the simulator developed during the
			preliminary interconnect study in \S\ref{sec:interconnect-modelling}.
			Like INSEE, it has been already been used to simulate the SpiNNaker
			Interconnect and so configurations exist for SpiNNaker like machines.
			Because of the author's familiarity with the tool, extending the router
			model should be straight-forward.
	
	
	\section{Small-World Network Experiments}
		
		With the extended and proven simulator developed, the next stage of my
		research will be to use it to model the behaviour of small-world style
		networks with more realistic traffic and wiring constraints.
		
		The work done in \S\ref{sec:small-world-super-computers} measured average
		shortest-path length in the networks examined. This measure corresponds to
		the average path length for uniform-random traffic in a real network with
		the same topology. In brain simulation systems such as spinnaker, much of
		the traffic is local and so uniform-random traffic is not representative.
		This work will test the performance of small-world networks with more
		realistic traffic.
		
		The other shortcoming found during the preliminary study was that when
		physical wiring was restricted in length (as in real systems), the logical
		distance covered by links in different parts of the system becomes uneven.
		Since the folding and rack/cabinet placement which takes place in real
		machines such as SpiNNaker is much more intricate than the simple torus
		studied, these effects may become insignificant.
		
		This work is intended to take around one further month including ample time
		for writing up and further experimentation with wiring schemes if required.
	
	
	\section{Topology Comparison}
		
		% Look at a whole load of popular topologies and possibly look at having
		% two/more networks.
		
		With the small-world network testing complete, the simulator should now be
		mature enough to begin further experiments with alternative, potentially
		more structured topologies. This will be done to empirically test
		alternative topologies for use in SNN simulation.
		
		This work will follow on from the experiments carried out by
		\cite{vainbrand11} which was limited to widely used, more generic network
		topologies with alternatives which may be better suited to the problem. For
		example, topologies such as express cubes \cite{dally91} offer improved
		performance for a limited amount of non-local traffic while preserving the
		local performance of mesh and torus networks.
	
	\section{Placement and Routing}
		
		Given the scale of the SpiNNaker machine and the networks it simulates, the
		NP-complete task of allocating processing tasks in each node in the system
		(placement) and routing messages between them (routing) is an important
		consideration. Current architectures, such as SpiNNaker, use table based
		routing where a lookup table is used to decide where to route messages
		arriving at each node in the system. Such systems offer a lot of flexibility
		in the routing schemes that can be implemented but also may introduce
		constraints on the complexity of routes through the system when the routing
		tables are restricted in size.
		
		The use of alternative topologies will require alternative routing schemes
		to be specified. In the above comparative work on possible topologies
		simple, na\"ive routing algorithms will be used. These simple to implement
		algorithms often result in non-optimal behaviour, especially with regard to
		load balancing\cite{dally04}. Given that systems such as SpiNNaker use
		packet-dropping flow control, where congested routes can cause packets to be
		dropped, the effects of load imbalance may cause localised areas of high
		packet losses which may significantly affect simulations.
		\cite{greenfield10}.
		
		In this phase of work alternative routing algorithms which attempt to handle
		routing table depth and congestion control will be studied. The former
		problem is similar in style to the problem faced by chip and printed circuit
		design automation tools. The latter side is covered largely by more
		conventional interconnect/network routing algorithms. Combining and
		evaluating these two approaches will be where much of this time is spent.
	
	\section{Effects of Multicast}
		
		Following on from work on basic routing, the challenge of performing routing
		for multicast traffic must be considered. Multicast routing must route a
		message from a single source to multiple destinations, in the case of
		spiking neural networks, potentially thousands of destinations.
		
		Current approaches to multicast routing again are split between chip and
		circuit design and interconnect/network routing. Chip designs often
		constrain the paths in various ways, for example to ensure that latency to
		each destination is similar. This parallels the need to try and control
		branching to minimise routing table resource. Network designs, however, are
		often designed to reduce the amount of bandwidth consumed by branching as
		late as possible. Once again, work must be done to combine and compare these
		two approaches.
	
	\section{Interconnect Technology Evaluation}
		
		% Can't use Silistix any more, HSS is nice but should we be using it
		% on-chip?
		
		Given a suitable topology, a further question is that of the actual
		implementation of the interconnect. Historically small-scale systems have
		used synchronous signals to communicate between various parts of the system.
		As systems have scaled, this is no longer feasible as clock distribution
		grows ever more difficult. Instead GALS architectures have become prominent
		where individual, synchronous components communicate via an asynchronous
		medium.
		
		The SpiNNaker chip uses an asynchronous NoC based on CHAIN
		\cite{plana07,bainbridge02} using IP from (the now defunct) Silistix Ltd.
		This interconnect is able to handle arbitration between the many incoming
		signals from on board processors and external connections to other chips.
		Since this technology is no-longer available, alternatives must be sought.
		
		SpiNNaker chips are interconnected using parallel, delay insensitive
		interconnects. These links, while adequate for the system's current scale,
		may not be trivially sped up. More suitable external interconnects will
		likely be based on high speed serial as described in
		\S\ref{sec:high-speed-serial}.
		
		Work must be done to evaluate what technologies will be best suited for us
		in a future system.
	
	\section{`SpiNNaker 2' Architecture}
		
		It is likely that a new version of SpiNNaker will be designed with some
		funding coming in soon. Based on the research into the interconnect here,
		would an alternative topology be appropriate?
		
		\subsection{Types of Transmission}
			
			Switching away from Silistix (they went bye-bye!), alternative
			technologies, e.g. high-speed serial, might be used to link chips
			together. Such technologies may have very different cost structures and
			call for new network topologies. Given we currently have concentrated
			multiplexed links, why not have the same for stuff lower down?
		
		\subsection{Multiple Networks}
			
			SpiNNaker's interconnect is obviously heavily targeted at neural
			simulation but at the expense of other activities. For example, set-up and
			management are a bit of a pain. Also, other algorithms don't suit it so
			well. It is possible SpiNNaker will benefit from having a second network
			for these purposes?
			
			What about time-division-multiplexing two different protocols onto one
			high speed serial link?

