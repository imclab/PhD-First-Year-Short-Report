\chapter{Preliminary Work}
	
	\section{High-Speed Serial Link Modelling}
		
		In order to understand the effects of inserting non-uniformity to the
		SpiNNaker Interconnect caused by the high speed links.
		
		\subsection{Objectives}
			Try and find out what is going on.
		
		\subsection{Results}
			Found a significant increase in average latency. Maybe this could be
			overcome by making more use of that added processing time by jumping on
			further ahead? Further simulation needs to be done anyhow...
	
	\section{SpiNNaker PCI-Express Interface}
		
		One way of getting data from one part of the machine to the other would be
		not to just dump it through another link but to pull it straight out into a
		conventional computer via PCI-Express and then maybe re-route it with
		fancier software or via the web.
		
		\subsection{PCI-Express}
			
			What is PCI-Express? How does it work? Where is it found?
		
		\subsection{High Speed Serial on FPGA}
			
			What is an FPGA? We have one on the boards. FPGAs contain hard-wired
			blocks which do special purpose things, one such job is to do PCI-Express.
	
	\section{Wiring-Up Super Computers}
		
		A practical constraint on any interconnect is that it should be possible to
		wire it up. In particular constraints exist on both on wire length and the
		practical difficulty of connecting up the wires into the correct places.
		Computers usually placed in racks. Tool can be used to study wiring
		constraints of new links.
		
		\subsection{Wire-Length Limits}
			
			What type of signalling is possible? Why are long wires bad? How long is
			long? Why not just the average wire length?
		
		\subsection{Topology}
			
			SpiNNaker has three dimensions of wiring connected in a torus. This has
			long wires. A torus intuitively has long wires along the way which causes
			a bit of a headache. You can slice it to make it rectangular, you can fold
			it to make the wires short (must fold 4x2) and divvy up into cabinets..
		
		\subsection{\emph{SpiNNer} Wiring Guide Generator}
			
			A tool was built, manipulates nodes which are pre-connected with the
			correct wiring graph. Produces instructions.
		
		\subsection{Further Work}
			
			Simplify instructions, alternative wirings.
	
	\section{Small-World Super Computers}
		
		In the same way that random search is generic (no free lunch), I tried
		adding random wires. This is backed up by Watts and Strogatz's small-world
		network producing algorithm. Built a model. Took into account wiring using
		techniques in Wiring-Up Super Computers.
		
		\subsection{Results}
			
			Even with wire-length limits in place, when folded the effect is still
			noticeable.
		
		\subsection{Further Work}
			
			Must try this with SpiNNaker topology and racks (maybe even better). Would
			like to formalise the result into a formula.
	
	\section{Place and Route}
		
		Tests written for the Place and Route system for SpiNNaker-103. Gained some
		experience with the task. 
		
		\subsection{Potential Improvements}
			
			The algorithms used for placement and routing are naive: generally greedy
			algorithms.



