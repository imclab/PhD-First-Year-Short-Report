\documentclass[a4paper,12pt,titlepage]{report}

\usepackage{amsmath}
\usepackage{fullpage}
\usepackage{tikz}

\title{Improving the Interconnection Network of a Brain Simulator}
\author{Jonathan Heathcote}
\date{The University of Manchester}

% Number subsubsections
\setcounter{secnumdepth}{3}

\begin{document}
	
	\maketitle
	
	\begin{abstract}
		
		The human brain is one of the most complex structures known to man and
		allows us to perform remarkable range of computational feats completely
		inaccessible to any machine. Recent efforts to study the brain have led
		researchers to devise novel massively-parallel computer architectures in
		order to cope with the unique demands of brain simulation software. The
		SpiNNaker project is developing one such machine based on over 1,000,000
		custom low-power ARM processors. My research aims to improve the way in
		which these processors connect together to improve the performance of
		simulations and other software on the system.
		
	\end{abstract}
	
	% XXX: Include subsubsections
	\setcounter{tocdepth}{4}
	\tableofcontents
	
	
	
	\chapter{Introduction}
		
		\section{Parallel Computing}
			
			The race to drive up the speed of individual processor cores ground to a
			halt in the mid 2000s leaving designers pushed to find a use for the vast
			number of transistors made available by Moore's law. In the years that
			followed we've seen multi-core processors appearing in everything from
			mobile phones to large super computers.
				
			\subsection{Programming}
			
			\subsection{Architecture}
		
		\section{Computational Problems}
			
			\subsection{TODO: More Examples}
			
			\subsection{Fluid Dynamics}
			
			\subsection{Brain Simulation}
				
				% TODO: Cite overview of computational neuroscience?
				
				The brain is an extremely powerful computer about which little is
				understood. The field of computational neuroscience hopes to bring
				understanding of the computational abilities and mechanisms of the
				brain.
				
				One approach to this problem is to try and produce simulations of models
				of the brain in order to understand and study their behaviour. These
				approaches often take the form of simulated populations of neurons, one
				of the basic building-blocks of the brain. Such models typically consist
				of a set of neurons, each connected to many other neurons.  Unlike
				digital circuits where each individual component connects to only a few
				or even one other component, neurons tend to be connected to hundreds or
				thousands of other neurons.
				
				Simulations of the brain therefore present a computational challenge as
				large amounts of communication must take place between all the neurons
				in the simulated system. Various architectures have been designed to
				solve this task and a selection are presented in Chapter
				\ref{chap:background}.
		
		\section{Architecture}
			
			\subsection{Topology}
			
			\subsection{Routing}
			
			\subsection{Multicast}
	
	
	
	\chapter{Background}
		\label{chap:background}
		
		\section{State of the Art}
		
			\subsubsection{Top 500}
			
			\subsubsection{Neural Simulators}
			
			\section{Spatial Computing}
				
				\subsection{Architectures}
					
					\subsubsection{Amorphous Computing}
					
					\subsubsection{Blob Computing}
				
				\subsection{Dynamic Interconnect}
				
				\subsection{Routing}
		
		\section{SpiNNaker}
			
			\subsection{Neural Networks}
			
			\subsection{Topology}
				
				\subsubsection{Hexagons}
				
				\subsubsection{Addressing}
			
			\subsection{Routing}
				
				\subsubsection{Dimension-Order Routing}
				
				\subsubsection{Multicast}
				
				\subsubsection{Table Based Routing}
			
			\subsection{Hardware Abstractions}
			
			\subsection{Spin-Link}
		
		
		\section{High Speed Serial}
	
	
	
	\chapter{Preliminary Work}
		
		\section{High-Speed Serial Link Modelling}
			
			\subsection{Objectives}
		
		\section{SpiNNaker PCI-Express Interface}
			
			\subsection{High Speed Serial on FPGA}
			
			\subsection{Applications}
		
		\section{Wiring-Up Super Computers}
			
			\subsection{Topology}
			
			\subsection{Wire-Length Limits}
			
			\subsection{\emph{SpiNNer} Wiring Guide Generator}
			
			\subsection{Further Work}
		
		\section{Small-World Super Computers}
			
			\subsection{Results}
			
			\subsection{Further Work}
		
		\section{Place and Route}
			
			\subsection{PACMAN}
	
	
	\chapter{Research Plan}
	
		\section{Short Term}
			
			\subsection{SpiNNaker Interconnect Model}
			
				\subsubsection{Performance Benchmarking}
				
				\subsubsection{Interconnect Experiments}
			
			\subsection{Effects of Multicast}
			
			\subsection{Routing in Hybrid Topologies}
		
		\section{Longer Term}
			
			\subsection{Further Network Topologies}
			
			\subsection{`SpiNNaker 2' Architecture}
				
				\subsubsection{Types of Transmission}
				
				\subsubsection{Multiple Networks}
	
	
	\chapter{Conclusion}
	
\end{document}
